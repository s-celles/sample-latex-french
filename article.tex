% Article en français avec bibliographie
% Compiler avec LuaLaTeX ou pdfLaTeX

\documentclass[a4paper, 11pt]{article}

% === ENCODAGE ET LANGUE ===
\usepackage[french]{babel}
\usepackage{csquotes}

% === MISE EN PAGE ===
\usepackage[
    top=2.5cm,
    bottom=2.5cm,
    left=2.5cm,
    right=2.5cm
]{geometry}

% === POLICES (décommenter pour LuaLaTeX) ===
% \usepackage{fontspec}
% \setmainfont{TeX Gyre Termes}

% === TYPOGRAPHIE ===
\usepackage{microtype}

% === MATHS ===
\usepackage{amsmath, amssymb, mathtools}
\usepackage{siunitx}
\sisetup{locale=FR}

% === TABLEAUX ET FIGURES ===
\usepackage{graphicx}
\usepackage{booktabs}
\usepackage{float}
\usepackage{caption}
\usepackage{subcaption}

% === BIBLIOGRAPHIE ===
\usepackage[
    backend=biber,
    style=numeric,          % ou authoryear, apa, ieee...
    sorting=nyt,
    maxbibnames=99
]{biblatex}
\addbibresource{references.bib}

% === LIENS ===
\usepackage[
    colorlinks=true,
    linkcolor=blue!60!black,
    citecolor=green!50!black,
    urlcolor=blue!70!black
]{hyperref}
\usepackage[french]{cleveref}

% === EN-TÊTES ===
\usepackage{fancyhdr}
\pagestyle{fancy}
\fancyhf{}
\fancyhead[L]{\leftmark}
\fancyhead[R]{\thepage}
\renewcommand{\headrulewidth}{0.4pt}

% === CODE SOURCE (optionnel) ===
\usepackage{listings}
\lstset{
    basicstyle=\ttfamily\small,
    numbers=left,
    numberstyle=\tiny,
    frame=single,
    breaklines=true
}

% === MÉTADONNÉES ===
\title{Titre de l'article}
\author{Prénom Nom\\
    \small Affiliation\\
    \small \texttt{email@example.com}
}
\date{\today}

% ============================================
\begin{document}

\maketitle

\begin{abstract}
    Résumé de l'article en quelques phrases. Cet article présente...
    
    \textbf{Mots-clés :} mot1, mot2, mot3
\end{abstract}

\tableofcontents
\newpage

% === INTRODUCTION ===
\section{Introduction}

Voici une citation~\cite{exemple2023}. On peut aussi citer plusieurs sources~\cite{exemple2023, autreexemple2022}.

Les guillemets français s'utilisent ainsi : \enquote{texte entre guillemets}.

% === MÉTHODES ===
\section{Méthodes}

\subsection{Sous-section exemple}

Une équation numérotée :
\begin{equation}
    E = mc^2
    \label{eq:einstein}
\end{equation}

On peut y faire référence avec \cref{eq:einstein}.

Un tableau propre :
\begin{table}[htbp]
    \centering
    \caption{Résultats expérimentaux}
    \label{tab:resultats}
    \begin{tabular}{lcc}
        \toprule
        Paramètre & Valeur & Unité \\
        \midrule
        Température & \num{25.3} & \si{\celsius} \\
        Pression & \num{101325} & \si{\pascal} \\
        \bottomrule
    \end{tabular}
\end{table}

% === RÉSULTATS ===
\section{Résultats}

Une figure :
\begin{figure}[htbp]
    \centering
    % \includegraphics[width=0.8\textwidth]{image.png}
    \fbox{\parbox{0.6\textwidth}{\centering [Placeholder pour image]}}
    \caption{Description de la figure}
    \label{fig:exemple}
\end{figure}

Référence à la \cref{fig:exemple} et au \cref{tab:resultats}.

% === DISCUSSION ===
\section{Discussion}

Discussion des résultats...

% === CONCLUSION ===
\section{Conclusion}

Conclusion de l'article...

% === BIBLIOGRAPHIE ===
\printbibliography[heading=bibintoc, title={Références}]

% === ANNEXES (optionnel) ===
% \appendix
% \section{Données supplémentaires}

\end{document}
